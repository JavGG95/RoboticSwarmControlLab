\documentclass[letterpaper, 10 pt, conference]{ieeeconf}


\usepackage[colorlinks=true, urlcolor=blue, pdfborder={0 0 0}]{hyperref}
\usepackage{geometry}
\usepackage{overpic}
\graphicspath{{./pictures/pdf/},{./pictures/ps/},{./pictures/png/},{./pictures/jpg/}}
\begin{document}
\author{Aaron T. Becker}
\title{Particle Logic 2.0}
\maketitle

\begin{abstract}
This introduces the first \emph{pizza problem}, an challenge of unknown complexity that is accessible to all lab members.  A successful completion of one part of this problem entitles the solver and a guest to a free tutorial and meal at Prof. Becker's home -- learn how to make hand-tossed pizza!  
\end{abstract}

 In previous work \cite{Becker2013f,Becker2014,Becker2014a,Becker2015}, our lab showed the computational universality of swarms of micro- or nano-scale robots
in complex environments, controlled not by individual navigation, but by a uniform global, external force.
Consider a 2D grid world, in which all obstacles and robots are unit squares,
and for each actuation, robots move maximally until they collide with an
obstacle or another robot. We demonstrated components of \emph{particle computation} in this
world, designing obstacle configurations that implement {\sc and} and {\sc or}
logic gates:  by using dual-rail logic, we designed {\sc not, nor, nand, xor,
xnor} logic gates. With the help of 2$\times$1 robots we designed  {\sc fan-out} gates, which
is necessary for simulating the full range of complex interactions
that are present in arbitrary digital circuits.  Using these gates we are able to establish the full range
of computational universality as presented by complex digital circuits.  As
an example we connect our logic elements to produce a 3-bit counter.  We also
demonstrated how to implement a data storage element.

   \begin{figure}
   \centering
   \href{http://youtu.be/EJSv8ny31r8}{
\begin{overpic}[width =\columnwidth]{DSC_0093lowres.JPG}%\put(30,-7){ $m=1$, partition 1}
\end{overpic}}
\caption{
\label{fig:prototype}
Gravity-fed hardware implementation of  particle computation.  The reconfigurable prototype is setup as a {\sc fan-out} gate using a 2$\times$1 robot (white). This paper proves that such a gate is impossible using only 1$\times$1 robots. \href{http://youtu.be/EJSv8ny31r8}{See the demonstrations in the video attachment \url{http://youtu.be/EJSv8ny31r8}.} }
\vspace{-1em}
\end{figure}


The following are open projects for our lab:

\begin{enumerate}
\item our current memory element is not good enough for a journal submission. It is not conservative.  I'd like to represent state (true or false) by the position of a 2x1 slider.  This slider may require more than 1 CW cycle.

\item a collection of obstacles and 2x1 sliders that work as a one-cycle delay:\\
\begin{figure}
\begin{centering}
\frame{ reservoir full of 1x1 components} \\
$\downarrow$\\ 
\frame{connecting line that feeds 1x1 components }\\
$\downarrow$\\ 
\frame{   one-cycle-delay      }   \\
\end{centering}
\caption{
\label{fig:onccycledelay}
 a collection of obstacles and 2x1 sliders that work as a one-cycle delay }
\end{figure}

   
OUTPUT:  every other CW cycle a 1x1 block comes out.  Currently we get a new component every CW cycle.  We'd like to get one every other cycle, or concatenate these to get a 1x1 output every n cycles.

\item \emph{Preferential Assembly}
  Add tiny magnets to some of the edges of the sliders so that sliders preferentially assemble if two sliders with mating arrangements hit each other. 
  \begin{itemize}
  	\item how many \emph{species} or \emph{slider varieties} can we construct?
	\item our round sliders rotate, and don't have preferential orientations.  We could weight one side of the sliders.   Could octagonal sliders work in our present system?  Would they rotate?
	\end{itemize} 
\item Develop theory showing that we can use a CW cycle and deliver a polyomino part to a destination, and then with another CW cycle hit this part at an arbitrary location with an additional 1x1 part.  I want to show that we can build arbitrary shaped polyominoes, or prove that certain shapes cannot be constructed. For instance, using our sliders with magnets, can we construct an `o'-shaped arrangement using eight 1x1 sliders?
\item read about \href{http://arxiv.org/abs/1408.3351}{Sandor's aTAM theory} and see if anything practical can be made of it.
\end{enumerate}


 \bibliographystyle{IEEEtran}
\bibliography{IEEEabrv,..//bib/aaronrefs}%,../aaronrefs} %../../../../../../ensemble/bib/aaronrefs}


 
 \end{document}